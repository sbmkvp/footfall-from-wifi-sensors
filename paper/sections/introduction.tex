New and developing ‘smart’ technologies today provide the infrastructure over
which movements and interactions of people can be measured and monitored in the
‘sentient city’ \citep{amin2017seeing}. This is making it possible to
reinvigorate conceptualise the city as the locus of human activities
supplementing night time geographies of residence \citep{martin2015developing}
with geographies of shopping behaviour \citep{lloyd2018detecting}, workzone
geographies \citep{singleton2018london} and studies of movement trajectories
\citep{campbell2008transforming}. This is rendering activity-based conceptions
of human behaviour central to analysis of hardship and opportunity in and
around the smart city \citep{venerandi2015measuring}.

Sentient technologies include mobile phone networks, which can triangulate user
locations relative to networks of masts, use of GPS to locate users of social
media services, and Wi-Fi access points connectivity to access the Internet.
These technologies offer differing levels of spatial precision, where mobile
telephony and Wi-Fi generally being less reliable and offer lower precision
than GPS to the end users, while being more advantageous for broader mobility
studies \citep{pinelli2015comparing}. There has been considerable research into
the usefulness of these technologies to understand patterns of movement in
cities in near real time \citep{candia2008uncovering,
gonzalez2008understanding, calabrese2013understanding}.  Most of this research
has focused upon technical specification of accuracy or precision
\citep{song2010limits,lane2010survey}, with somewhat less attention devoted to
the ways that the characteristics of the technologies and of their human users
conspire to create possible bias in representing usage patterns across the
entire smart city. 

Even analysis of mobile phone data, usually derived from industry players that
have significant market share and user bases representative of local
populations, may exclude groups such as tourists from distant origins or
subscribers to third party services that share distinctive characteristics
\citep{di2016mind}

These examples illustrate the issues that underpin the assembly and analysis of
consumer data, which can be considered as a distinctive class of Big Data that
arise from the interactions between humans and customer-facing organisations
such as retailers, domestic energy suppliers, transport providers and suppliers
of social media and communications \citep{cdrc2018consumer}.  Consumer data
account for an ever-increasing real share of all of the data that are collected
of citizens, but a fundamental characteristic of consumer-led markets is that
no single provider has a monopoly in market provision, and therefore issues of
market share and segment generate bias in analysis.  The source and operation
of this bias is unknown in the absence of extensive and context sensitive
attempts to triangulate consumer data with data of known provenance relating to
clearly defined populations \citep{lansley2016deriving}.  In similar ways to
other classes of Big Data, consumer data are best thought of as digital
‘exhaust’, or a by-product created by or harvested from consumer transactions.

In this paper, through a set of experiments, we evaluate the value of data
collected from a Great Britain wide network of 800 devices \citep{sss2016}
installed in order to characterize the footfall patterns of a scientifically
balanced sample of retail centres.  These devices are located in shop windows
and record the probes emitted by mobile phones and other Wi-Fi enabled devices.
The data collected from these devices are deemed to be consumer data because
devices carried by consumers routinely probe for Wi-Fi connection which is a
consumer service.  Monitoring the probes from such devices provides an
indication of the presence of their users, regardless of whether or not
internet connectivity is established.  Our core motivation is to appraise the
usefulness of Wi-Fi probe requests harvested from our network of sensors in
order to indicate levels of pedestrian activity.  More broadly still, in our
future research we intend to classify the nationwide network of footfall
profiles as part of a programme of research to understand the form and
functioning of retail areas at a time of far-reaching structural change for the
retail industry.

To this end, it is important to first undertake a thorough conceptual and
technical appraisal of our consumer data source.  In technical terms, screening
the information present in the `probes requests' and clustering them based on
their characteristics is essential in order to remove the ones emitted by
devices that do not indicate pedestrian activity, such as network enabled
printers and other fixed devices.  Related to this, a method to fingerprint
Wi-Fi probes is necessary to remove probes from individuals’ devices that in
conceptual terms should not be considered part of footfall – as when, for
example, an employees is seated in an office within range of the sensor device.
A calibration of sensor measurement is also essential on two grounds: first
individuals may carry multiple devices, or no device at all; and second, the
positioning and orientation of the sensor in the retail unit may lead to
systematic over- or under-enumeration.  These sources of bias in measurement
must be accommodated by manual recording of footfall at each location and
generalization of these sample survey results to all locations and time
periods.  As we describe in detail below, manual validation of the data needs
to be undertaken in parallel with technical profiling of the mix of consumer
mobile devices that probe our sensors, since the effectiveness of data cleaning
procedures discussed in this paper differ between individual locations and
configurations.
