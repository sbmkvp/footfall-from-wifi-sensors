In the past decade Wi-Fi has emerged as the most commonly used technology in providing high speed internet access to mobile devices such as smartphones, tablets and laptops in public and private spaces. This has resulted in multiple Wi-Fi networks being available at almost every location in dense urban environments. Traversing through this overlapping mesh of Wi-Fi networks, modern mobile devices with Wi-Fi antennae regularly broadcast a special type of signal known as 'Probe Requests', in order to discover Wi-Fi networks available to them. This helps these devices to connect and switch between the WiFi networks seamlessly.

Probe requests are low level signals standardised by IEEE 802.1b/g specification \citep{ieee2013} as the first step in establishing a Wi-Fi based connection between two devices and is implemented in any Wi-Fi capable device irrespective of the manufacturer or the model. This ubiquity and standardisation make them an excellent source of open, passive, continuous, and wireless data generated by Wi-Fi capable devices present at any given time and location. Considering the unprecendented levels of mobile device ownership in recent years, we can in turn use this data to understand the population distribution in highly dynamic urban environments with high spatial and temporal granularity \citep{freud2015,konto2017}.

While a Wi-Fi based method to collect data offers us various advantages such as, easy scalability and efficiency in terms of cost and time, It also introduces few systematic biases, uncertainities in the collected data along with the serious risk of infringing on the privacy of the mobile users. In this paper, using a set of probe requests and manual counts collected at various high street locations across London, we demonstrate that pedestrian footfall at these locations can be estimated with considerable precision and accuracy while protecting the privacy of the pedestrians.
