This paper has made a novel contribution to the measurement of activities in retail centres in real time.
To this end, it has described the collection and processing of a novel consumer Big Dataset that enables valid measures of levels of footfall activity which has been scaled across a Great Britain wide network of sensors.
In both conceptual and technical terms, it illustrates the ways in which passively collected consumer data can be ‘hardened’ to render them robust and reliable by using related procedures of internal and external validation.

Internal validation addresses the issues of screening out device probes that do not indicate footfall, and the further screening of device probes to ‘fingerprint’ the effects of MAC randomization.It is important to note that the filtering process work based soley on the information present in the probe requests and their temporal distribution.
This ensures that although the mobile devices were uniquely identified, there was no further personal data generated by linking the probe requests to the users of the mobile devices.
This method essentially gave us a way to estimate the footfall in real-time without identifying or tracking the mobile devices themselves. 
External validation then entails reconciling adjusted counts with the footfall observed at sample locations.
This procedure makes it possible to generalise from locations at which manual footfall surveys are conducted to all others in the system, and to develop a classification of device locations that are more or less susceptible to noise generation.


This Wi-Fi based footfall counting methodology offers a large number of applications and benefits for real time spatial analysis.
Since Wi-Fi based sensors are inexpensive and the data model is scalable, it is possible to use this methodology for a large network of sensors to gather granular data on pedestrian footfall.
A snapshot showing week's worth of precise footfall in area around Charring cross, London is shown in Figure \ref{} to demonstrate the potential for such dataset.
Projects such as SmartStreetSensors \citep{sss2016}, may utilise this methodology to overcome the challenges introduced by the implementation of MAC address randomisation.
Such precise and granular data also enables us to confidently model the pedestrian flow in urban road networks, and will be an indispensable tool in the smart city framework.
It can also be used to understand and classify geographical areas based on the spatio-temporal distribution of the volume of activity in them which we intend to research in future.

