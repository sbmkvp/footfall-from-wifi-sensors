The aims of the main study are,
test the validity of the signal strength algorithm in different 
micro site conditions.
Test that the sequence number algorithm works in real world
for different locations and different times.
Test if the calibration works over intervals
Finally conclude if we can estimate footfall confidently
with just probe requests.

\subsection{locations}
five locations were selected across central london which
had different types of configurations and specific problems
configuratons are shown in fig. map is shown in figure.
\begin{enumerate}
	\item Phone Shop Camden - has phones and bus stops.
	\item Restaurant TCR - has seating area on either side.
	\item Holborn Information Kiosk - High volumne station entry
	\item Restaurant Russell Square - seating on one side and side walk on other
	\item Shop Charring Cross - sidewalk on one side and phone shop next door
\end{enumerate}
installations were carried out over the time period from xxxx to xxxx.
the data collection happened from xxxx to xxxx. Manual counting was carried out
with high precision on dates xxxx and aggregated five minutes on xxxx.
The difference in methods could lead to some inaccuracies in data.
The overall statistics of data collected. 
The overally schedule is shown in half page graphic.

\subsection{Signal strength filtering}

\subsection{Device Fingerprinting}

\subsection{Manual Calibration}

\subsection{Discussion}
