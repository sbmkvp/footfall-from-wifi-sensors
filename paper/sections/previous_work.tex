Though WiFi is a `location-less' technology, there are reliable methods to triangulate the location of Wi-Fi enabled mobile devices by the known locations of APs and the signal strength reported by them from the mobile devices \citep{he2003range, moore2004robust, lamarca2005place}.
This can overcome the usual shortcoming of GPS, which struggles for precision and accuracy in indoor and densely built environments \citep{zarim2006,kawaguchi2009wifi, xi2010locating}. 
Utilising this, we can easily and quickly estimate trajectories of the mobile devices just using the WiFi communication the device has with multiple known APs \citep{Sorensenlocation} which can be used similar to the GPS trajectories to understand individual travel patterns (Kim, 2006;\citep{reki2007,Sap2015}, crowd behaviour \citep{abedi2013bluetooth,mowafi2013tracking}, vehicular \citep{lu2010vehicle} and pedestrian movement \citep{xu2013pedestrian,fukuzaki2014pedestrian,wang2016gait}.
Such data can also be used in transportation planning and management to estimate travel time \citep{musa2011wiflow} and real time traffic monitoring \citep{abbott2013empirical}.

Using techniques demonstrated by \cite{franklin2006passive} and \cite{pang2007802} and globally unique information present in the probe requests one can also successfully track people across these access points \citep{cunche2014linking}, their trajectories \citep{musa2012tracking}, and the interactions between them \citep{cheng2012inferring,barbera2013signals,cunche2014know} such as predicting which of them are most likely to meet again \citep{cunche2012know}.
Using the semantic information present in these probe requests it even is possible to understand the nature of population at a large scale \citep{di2016mind}. 

Though extensive research has been done on this subject with feasible and favorable results, in recent years, one of the major challenges faced in such attempts has been the increasing attempt by mobile phone manufacturers to protect the users’ privacy by anonymising the globally identifiable portion of the probe requests, \citep{green2008}.
There are various methods which have been devised to overcome this anonymisation process such as decomposition of OUIs where detailed device model information is estimated by analysing an already known dataset of OUIs \citep{martin2016decomposition}; Scrambler attack using a small part of the physical layer specification for WiFi \citep{vo2016,bloessl2015scrambler}; and finally, the timing attack where the packet sequence information present in the probe request frame is used \citep{matte2016,cheng2016can}.
A combination of these methodologies has been proven to produce de-anonymised globally unique device information \citep{vanhoef2016, martin2017}. These approaches usually result in serious risk of infringement of the privacy of the users of the mobile devices by revealing their identifiable personal information.

There is a clear gap in the research for exploring methodologies which enable us to estimate the number of unique mobile devices from a set of anonymised probe requests, without the need to reveal their original MAC addresses.
Such technique has various applications in numerous fields such as uncovering the urban wireless landscape \citep{rose2010mapping}, revealing human activity at large scales \citep{qin2013discovering}, estimating pedestrian numbers in crowds \citep{schauer2014estimating,fukuzaki2015statistical} and even counting people in hyper local scales such as queues \citep{wang2013measuring}
With enough infrastructure to collect such information we can aim to generate a real-time census of the city \citep{konto2017}.
It has also been demonstrated by \citep{pinelli2015comparing} through series of experiments on a telecom operator dataset that using such network-driven approach is more advantageous compared to the widely used event-driven approaches.

With this background we set out to device and implement a methodology to reliably estimate human activity such as pedestrian footfall from Wi-Fi probe requests without infringing the privacy of the users involve.
