% ==============================================================================
% DOCUMENT HEADER
% ==============================================================================

\documentclass[11t, a4paper, twocolumn]{article} 
\usepackage[english]{babel}
\usepackage{microtype}
\usepackage{amsmath,amsfonts,amsthm}
\usepackage[svgnames]{xcolor}
\usepackage[hang, small, labelfont=bf, up, textfont=it]{caption}
\usepackage{booktabs}
\usepackage{lastpage}
\usepackage{graphicx}
\usepackage{enumitem}
\setlist{noitemsep}
\usepackage{sectsty}
\allsectionsfont{\usefont{OT1}{phv}{b}{n}}
\usepackage{lipsum}
\usepackage{geometry}
\geometry{
	top=1cm,
	bottom=1.5cm,
	left=2cm,
	right=2cm,
	includehead,
	includefoot
}
\setlength{\columnsep}{7mm}
\usepackage[T1]{fontenc}
\usepackage[utf8]{inputenc}
\usepackage{XCharter}
\usepackage{fancyhdr}
\pagestyle{fancy}
\renewcommand{\headrulewidth}{0.0pt}
\renewcommand{\footrulewidth}{0.25pt}
\renewcommand{\sectionmark}[1]{\markboth{#1}{}}
\lhead{}
\chead{\textit{\thetitle}}
\rhead{}
\lfoot{}
\cfoot{}
\rfoot{\footnotesize Page \thepage\ of \pageref{LastPage}}
\fancypagestyle{firstpage}{
	\fancyhf{}
	\renewcommand{\footrulewidth}{0pt}
}
\newcommand{\authorstyle}[1]{{\large\usefont{OT1}{phv}{b}{n}\color{Black}#1}}
\newcommand{\institution}[1]{{\footnotesize\usefont{OT1}{phv}{}{sl}\color{Black}#1}}
\usepackage{titling}
\newcommand{\HorRule}{\color{Black}\rule{\linewidth}{0.75pt}}
\pretitle{
	\vspace{-30pt}
	\HorRule\vspace{10pt}
	\fontsize{30}{34}\usefont{OT1}{phv}{b}{n}\selectfont
	\raggedright
	\color{Black}
}
\posttitle{\par\vskip 15pt}
\preauthor{}
\postauthor{
	\vspace{10pt}
	\par\HorRule
	\vspace{5pt}
}
\usepackage{lettrine}
\usepackage{fix-cm}
\newcommand{\initial}[1]{
	\lettrine[lines=3,findent=4pt,nindent=0pt]{
		\color{DarkGoldenrod}
		{#1}
	}{}
}
\usepackage{xstring}
\newcommand{\lettrineabstract}[1]{
	\StrLeft{#1}{1}[\firstletter]
	\initial{\firstletter}\textbf{\StrGobbleLeft{#1}{1}}
}
\usepackage[backend=bibtex,style=numeric,natbib=true]{biblatex}
\addbibresource{ref.bib}
\usepackage[autostyle=true]{csquotes}


\title{Estimating highstreet footfall trends by analysing the Wi-Fi probe
requests received at shop fronts}
\author{
	\authorstyle{
		Balamurugan Soundararaj\textsuperscript{1}, 
		James Cheshire\textsuperscript{1} and 
		Paul Longley\textsuperscript{1}}
	\newline\newline
	\textsuperscript{1}\institution{
		Department of Geography, 
		University College London, 
		United Kingdom}
}
\date{\today}

% ==============================================================================
% DOCUMENT BODY
% ==============================================================================

\begin{document}

	% ---------------------------------------------------------------------------
	% We talk about wifi and probe requests. Possibility of 
	% counting people without revealing identity. There is uncertainity. We look
	% at resolving these. We have a setup to collect data from sensor and manual
	% We propose methodology to clean sensor count. compare with manual to see
	% it corresponds properly.
	% ---------------------------------------------------------------------------

	\maketitle
	\thispagestyle{firstpage}

	\textbf{Abstract : }
	Wifi has become an ubiquitous technology in provision of private and public 
	internet access to mobile devices such as smartphones,tablets and laptops.
	In addition to internet these devices use these networks as a way to quick 
	geo location without waiting for GPS.
	To produce a list of WiFi networks available, mobile devices constantly 
	broadcast signals called probe requests.
	These signals have various information regarding the device and its 
	capabilities which are identifiable such as MAC addresses, non identifiable 
	such as signal strength and partly identifiable such as randomised mac 
	addresses.
	There is a significant use in uniquely identify the number of mobile devices
	at a specific location wihout actually affecting the privacy of the mobile
	devices.
	Such enumeration can be done by looking at identifiable information after 
	anonymising them whenever they are available and from the patterns in the 
	other information when they are not.
	We collect data on probe requests sent at set of locations in 
	different times and device a methodology for enumerating number of people
	at these locations from the data.
	We compare the enumerated counts with the counts collected manually at
	these locations to find that we are able to estimate real life with a 
	confidence of XX\%.
	
	% ---------------------------------------------------------------------------
	% This section we discuss overall things. How passive data collection can be
	% boon in urban informatics.
	% The disadvantages for doing so and the need to overcome these.
	% Wifi - justify as good technology - advantages and disadvantages. Mention
	% LDC project and importance of doing so.
	% We briefly introduce what has been discussed in the following chapters.
	% ---------------------------------------------------------------------------

	\section{Introduction}\label{intro}

		
	\section{Previous Work}\label{prev}
		\lipsum[1-2]
		\citep{vanhoef2016}
		\citep{matte2016}
		\citep{martin2017}
		\citep{vo2016}
		\citep{konto2017}
	\section{Data Collection}\label{data}
		\lipsum[1-2]
	\section{Methodology}\label{method}
		\lipsum[1]
		\subsection{Vendor OUIs}
			\lipsum[2]
		\subsection{Frame length}
			\lipsum[2]
		\subsection{Sequence Numbers}
			\lipsum[2]

	\section{Results}\label{res}
		\lipsum[1]
	\section{Conclusions}\label{con}
		\lipsum[1-2]

	\printbibliography[title={References}]

\end{document}
