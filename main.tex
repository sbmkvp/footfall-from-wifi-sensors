\documentclass[11t, a4paper, twocolumn]{article} 
\usepackage[english]{babel}
\usepackage{microtype}
\usepackage{amsmath,amsfonts,amsthm}
\usepackage[svgnames]{xcolor}
\usepackage[hang, small, labelfont=bf, up, textfont=it]{caption}
\usepackage{booktabs}
\usepackage{lastpage}
\usepackage{graphicx}
\usepackage{enumitem}
\setlist{noitemsep}
\usepackage{sectsty}
\allsectionsfont{\usefont{OT1}{phv}{b}{n}}
\usepackage{lipsum}
\usepackage{geometry}
\geometry{
	top=1cm,
	bottom=1.5cm,
	left=2cm,
	right=2cm,
	includehead,
	includefoot
}
\setlength{\columnsep}{7mm}
\usepackage[T1]{fontenc}
\usepackage[utf8]{inputenc}
\usepackage{XCharter}
\usepackage{fancyhdr}
\pagestyle{fancy}
\renewcommand{\headrulewidth}{0.0pt}
\renewcommand{\footrulewidth}{0.25pt}
\renewcommand{\sectionmark}[1]{\markboth{#1}{}}
\lhead{}
\chead{\textit{\thetitle}}
\rhead{}
\lfoot{}
\cfoot{}
\rfoot{\footnotesize Page \thepage\ of \pageref{LastPage}}
\fancypagestyle{firstpage}{
	\fancyhf{}
	\renewcommand{\footrulewidth}{0pt}
}
\newcommand{\authorstyle}[1]{{\large\usefont{OT1}{phv}{b}{n}\color{Black}#1}}
\newcommand{\institution}[1]{{\footnotesize\usefont{OT1}{phv}{}{sl}\color{Black}#1}}
\usepackage{titling}
\newcommand{\HorRule}{\color{Black}\rule{\linewidth}{0.75pt}}
\pretitle{
	\vspace{-30pt}
	\HorRule\vspace{10pt}
	\fontsize{30}{34}\usefont{OT1}{phv}{b}{n}\selectfont
	\raggedright
	\color{Black}
}
\posttitle{\par\vskip 15pt}
\preauthor{}
\postauthor{
	\vspace{10pt}
	\par\HorRule
	\vspace{5pt}
}
\usepackage{lettrine}
\usepackage{fix-cm}
\newcommand{\initial}[1]{
	\lettrine[lines=3,findent=4pt,nindent=0pt]{
		\color{DarkGoldenrod}
		{#1}
	}{}
}
\usepackage{xstring}
\newcommand{\lettrineabstract}[1]{
	\StrLeft{#1}{1}[\firstletter]
	\initial{\firstletter}\textbf{\StrGobbleLeft{#1}{1}}
}
\usepackage[backend=bibtex,style=numeric,natbib=true]{biblatex}
\addbibresource{ref.bib}
\usepackage[autostyle=true]{csquotes}


\title{Enumerating unique mobile devices from a set of Wi-Fi probe requests}
\author{
	\authorstyle{Balamurugan Soundararaj\textsuperscript{1}, James Cheshire\textsuperscript{1} and Paul Longley\textsuperscript{1}}
	\newline\newline
	\textsuperscript{1}\institution{Department of Geography, University College London, United Kingdom}
}
\date{\today}

\begin{document}

	\maketitle
	\thispagestyle{firstpage}
	\textbf{Abstract }
	Wifi has become an ubiquitous technology. The service provision coverage is almost everywhere and almost all devices have wifi capabiliy for internet access and quick location. Probe requests are low level management packets used by devices defined by protocol. At any given location since people have mobile devices on them a host of wifi probe requests are sent. They are either identifiable or obfuscated. But still has patterns in them. We look into a method to convert these bunch of probe requests into a measure of number of people at that location. We start by looking into what is present in a probe request and how it can be used to count people anonymously.
	\section{Introduction}\label{intro}
		
	\section{Previous Work}\label{prev}
		\lipsum[1-2]
		\citep{vanhoef2016}
		\citep{matte2016}
		\citep{martin2017}
		\citep{vo2016}
		\citep{konto2017}
	\section{Data Collection}\label{data}
		\lipsum[1-2]
	\section{Methodology}\label{method}
		\lipsum[1]
		\subsection{Vendor OUIs}
			\lipsum[2]
		\subsection{Frame length}
			\lipsum[2]
		\subsection{Sequence Numbers}
			\lipsum[2]

	\section{Results}\label{res}
		\lipsum[1]
	\section{Conclusions}\label{con}
		\lipsum[1-2]

	\printbibliography[title={References}]

\end{document}
